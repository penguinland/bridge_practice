\usepackage{color}
\usepackage{fullpage}
\usepackage{longtable}

% Solid heart and diamond symbols for use with color fonts. Solution from
% https://tex.stackexchange.com/questions/9641/filled-diamondsuit-and-heartsuit/9643#9643
\DeclareSymbolFont{extraup}{U}{zavm}{m}{n}
\DeclareMathSymbol{\varheart}{\mathalpha}{extraup}{86}
\DeclareMathSymbol{\vardiamond}{\mathalpha}{extraup}{87}

% The rest is adapted from the bridge-pln package on CTAN:
% https://ctan.org/pkg/bridge-pln

% Suit definitions
\newcommand{\s}{\ensuremath{\spadesuit}}
\newcommand{\h}{\ensuremath{{\color{red}\varheart}}}
\renewcommand{\d}{\ensuremath{{\color{red}\vardiamond}}}
\renewcommand{\c}{\ensuremath{\clubsuit}}
\newcommand{\nt}{\textsc{nt}}

\newcommand{\hand}[4]{{%
  %Example: \hand{AKJ765}{AK9}{--}{T983}
  \setlength{\tabcolsep}{0.5ex}
  \begin{tabular}[t]{rl}
    \s&#1\\
    \h&#2\\
    \d&#3\\
    \c&#4\\
  \end{tabular}
}}

\newcommand{\NESW}{{
  \setlength{\tabcolsep}{0mm}
  \begin{centering}
  \begin{tabular}[t]{|ccc|}
    \hline
    &N&\\
    ~W&&E~{}~\\
    &S&\\
    \hline
  \end{tabular}
  \end{centering}
}}

\newcounter{dealno}

\newcommand{\deal}{
  % For an explanation of the use of \expandafter, see the definition of
  % \problem below.
  \ifdefined\showsolutions
    \expandafter\fulldeal
  \else
    \expandafter\isolatedhand
  \fi
}

\newcommand{\fulldeal}[6]{
  % Arguments are: dealer, vulnerability, N E S W
  \begin{tabular}[t]{lll}
    \stepcounter{dealno}
    \textbf{Deal \arabic{dealno}}&#3&
    \begin{tabular}[t]{l}Dlr: #1\\ Vul: #2\end{tabular}\\
    #6&\NESW&#4\\
    &#5&
  \end{tabular}
}

\newcommand{\isolatedhand}[6]{
  % Arguments are: dealer, vulnerability, N E S W
  % We only want to show dealer, vulnerability, and the South hand
  \begin{tabular}{lll}
    \stepcounter{dealno}
    \textbf{Deal \arabic{dealno}}&
    \begin{tabular}[t]{l}
      Dlr: #1\\ Vul: #2
    \end{tabular}&
    #5
  \end{tabular}
}

\newcommand{\ouralert}[1]{% Only show the explanation in the solution.
    \ifdefined\showsolutions\footnote{#1}\fi%
}

\newcommand{\oppsalert}[1]{% Always show the explanation
    \footnote{#1}%
}

\newenvironment{bidding}{
  \begin{minipage}[t]{1.9in}
    \begin{tabular}{llll}%
      \emph{West}&
      \emph{North}&
      \emph{East}&
      \emph{South}\\}{
    \end{tabular}%
  \end{minipage}%
}

% It might be simplest to just check if \showsolutions is defined once at the
% very beginning and change the difinition of \problem in response to that. but
% this way, we can define it and undefine it in the middle of the document and
% get interesting results. This is particularly useful during testing and
% debugging.
\newcommand{\problem}{%
  \ifdefined\showsolutions
    % We use \expandafter to get rid of the \else...\fi before passing
    % arguments to the \problemwithsolution.
    \expandafter\problemwithsolution
  \else
    \expandafter\problemwithoutsolution
  \fi
}

\newcommand{\problemwithsolution}[5]{
  % Arguments are: deal, bidding, answer, explanation, debugging reference
  \hline\\
  \begin{tabular}[t]{c}
    #1\\\\#2
  \end{tabular}
  &
  \begin{minipage}[t]{3in}
    Answer: #3\\ % Extra separation between the short answer and the explanation

    #4\\ % TODO: Maybe just change the paragraph separation length?

    {\footnotesize Debug Ref.: #5}
  \end{minipage}
  \vspace*{0.5ex}\\
}

\newcommand{\problemwithoutsolution}[5]{
  % Arguments are: deal, bidding, answer, explanation, debugging reference
  % We surround the whole thing with blank lines to give the hand a little more
  % room to be considered on its own.
  \hline\\
  #1&#2\\
  \\
}

\newenvironment{problemset}{
  \begin{center}
    \begin{longtable}[H]{ll}
}{
      \hline
    \end{longtable}
  \end{center}
}
